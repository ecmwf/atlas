\chapter{Create Fields and Field Sets}
In this chapter, we show how to create fields and field sets
using \Atlas. Specifically, we outline how to create two simple 
fields and how to include them into a field set. These two fields
are standalone, thus not related to any grid - i.e. they are just 
defined as generic multidimensional arrays containing some values 
and a short description of what is stored inside them.
Successively, we introduce how to create two fields on a given 
grid and again how to add them to a field set.
As done for the other examples, we show both the C++ and Fortran 
versions.


\section{Standalone Fields and Field Sets}
\label{sect:standalone-fields}

\subsection{C++ version}

\section{Fortran version}
The \lista{code-fields-F} shows how to construct two standalone 
fields and encapsulate them into a field set. 
%
\lstinputlisting[caption=Generating two fields and encapsulating 
them into a FieldSet using Fortran, style=FStyle, label=code-fields-F]{fields.F90}
%
On the first few lines of the code, we define the variables
needed for this program. In particular, the \Atlas specific 
variables needed for this example are the \inltc{atlas\_Field}, 
\inltc{atlas\_FieldSet} and \inltc{atlas\_metadata} objects.

After having defined all the data needed for this example, 
we initialize the \Atlas library as usual and we define 
two fields, one called \inltc{pressureField} that, for 
instance, will contain the pressure and the other one 
called \inltc{windField} that will for example contain 
the velocity of the wind in two orthogonal directions.

How does the creation of a field work?\\
On lines 16 and 17 we can see the construction of the two fields.
We first need to specify the name of the field (in our case 'pressure' 
and 'wind'), then we need to specify the type of the data stored 
into the fields (in our case double precision numbers) and finally 
we need to provide the dimension of the field.
Note that we allow multidimensional fields up to 6 dimensions 
(this number can be extended if required)!
%
\begin{tipbox}
In a field we can only store numbers - no strings or characters!
In particular, we support integers, float types and double types.
\end{tipbox}
%
Once the fields are defined we need to initialize them 
and give them some values. This task can be achieved 
by using the code on lines 20 and 21, where we set 
initializes the two fields to the two pointers \inltc{pressure}
and \inltc{wind}, respectively. 
We successively prescribe some values to these two pointers 
(see lines 25 to 27). This step automatically updates what 
is stored in the two field objects, \inltc{pressureField} 
and \inltc{windField}.

The work for defining the two fields is almost completed.
We can add just one more little feature - one or more 
descriptors. This task is performed on lines 31 to 36, 
where we use the \inltc{metadata} object to set the units 
of our fields and retrieve them through the functions, 
\inltc{set} and \inltc{get}, respectively.

These two fields are fully functional and can be used 
for our specific application. However, we may want to 
encapsulate several fields into one object. This task 
can be achieved by using the object \inltc{atlas\_FieldSet}, 
as show on lines 39 to 41, where we define the field 
set and we add the two fields into it.

Any field can also be retrieved from a field set by 
using the code on lines 44 and 45, where we ask for 
the field 'pressure' and the field 'wind' to be retrieved.
%
\begin{notebox}
It is possible to retrieve a field from a field set 
either by using the name of the field or by using 
the number identifying it. In our example, \inltc{pressureField} 
assumes id=1 (since stored first), while \inltc{windField}
assumes id=2 (since stored second).
\end{notebox}
%
After having defined the field set we finally print some 
useful information regarding the fields (for the sake of 
brevity we print just some information regarding the \inltc{windField}
- the information regarding the \inltc{pressureField} can 
be obtained in an identical way).
In particular, we print the name, the size and the \inltc{metadata}
associated to the \inltc{windField} (see lines 48 to 50).
We then extract its rank, shape and dimensions in bytes
(see lines 51 to 54). We finally print type of the data
stored in the field (see lines 55 and 56).

On lines 59 to 61 we also print the values of one element 
per each field and we destroy the field objects on lines 
64 and 65 (thus releasing the memory).

Note that destroying the \inltc{atlas\_Field} objects 
is enough to also destroy \inltc{atlas\_FieldSet} object;
we need to explicitly destroy it as well to completely 
free the memory associated to all the objects defined 
in this example (see line 66).

It is now possible to run this simple program typing
the following text on the terminal
%
\begin{lstlisting}[style=BashStyle]
./atlas_f-fields
\end{lstlisting}
% 
This will produce the two fields described and a field 
set and will destroy them. It will also print to the 
screen some useful information regarding the fields.

You can now play with the code in \lista{code-fields-F} 
to generate as many fields/field sets as you want! 




\section{Fields on a given Grid}
\label{sect:grid-fields}
\subsection{C++ version}


\subsection{Fortran version}
The \lista{code-fields-F} shows how to construct one field
on a given grid. To see how to create a generic field and 
a field set, please refer to \sect{sect:standalone-fields}
above. 
%
\lstinputlisting[caption=Generating a field on a given grid 
using Fortran, style=FStyle, label=code-fields-F]{fields-on-grid.F90}
%
On the first few lines of the code, we define the variables
needed for this program. In particular, we define some constants 
needed for the function we are going to implement later in the 
code and we declare an \inltc{Atlas\_reducedGrid} object and 
an \inltc{atlas\_Field} object.

On line 23 we define the grid using a command-line key 
that can be specified by the user (see \chap{chap:grids}), 
while on lines 25 and 26, we initialize the pressure field.
From line 28 to line 45, we specify the a Gaussian-type 
(e.g. a hill) function on our grid (specifically, the 
field is defined on line 41).
We finally close the program outputting on the screen 
the memory footprint of the field just created and the 
values of the pressure field on 7 grid points (see lines 
47 to 52).
As usual we also need to free the memory calling the 
function \inltc{final} on the grid and field objects.

It is now possible to run this simple program typing
the following text on the terminal
%
\begin{lstlisting}[style=BashStyle]
./atlas_f-fields-on-grid --grid O1024
\end{lstlisting}
% 
This will produce a field (called pressure) defined 
on an octahedral grid that has the shape of a hill 
or Gaussian-type function.

You can now play with the command-line argument and 
generate different grids and see the impact on the 
memory footprint of the pressure field.