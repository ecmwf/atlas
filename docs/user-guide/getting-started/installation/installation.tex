\chapter{Download and installation}
\label{chap:installation}

\section{Download and requirements}
For download information please go to: \\
\url{https://software.ecmwf.int/wiki/display/ATLAS/Atlas}
 
\Atlas additionally requires the following third party libraries
to be successfully compiled:
%
\begin{itemize}
\item CMake: For use and installation see http://www.cmake.org/
\item ecbuild:
\item eckit: (with mpi support - meaning either eckit\_mpi 
or eckit\_mpistubs library)
\item python (only when Fortran bindings are required)
\end{itemize}
%
Also, the following recommended packages might be considered:
%
\begin{itemize}
\item MPI: Distributed memory parallelisation;
\item OpenMP: Shared memory parallelisation;
\item fctest: Unit testing for Fortran;
\item boost\_unit\_test: Unit testing for C++;
\end{itemize}
%
\Atlas has been tested and is known to work with the following compilers
%
\begin{itemize}
\item GCC 4.8.1
\item Intel 13.0.1, 14.0.1
\item CCE 8.2.7, 8.3.1
\end{itemize}
%




\section{Installation}
Once you have downloaded the library you can install it by using 
an out-of-source build/install based on CMake. In particular you 
can follow the steps below:
%
\begin{lstlisting}[style=BashStyle]
# Environment --- Edit as needed
ATLAS_SRC=$(pwd)
ATLAS_BUILD=build
ATLAS_INSTALL=$HOME/local
export CC=mpicc
export CXX=mpicxx

# 1. Create the build directory:
mkdir $ATLAS_BUILD
cd $ATLAS_BUILD

# 2. Run CMake
ecbuild $ATLAS_SRC -DCMAKE_INSTALL_PREFIX=$ATLAS_INSTALL

# 3. Compile / Install
make -j10
make install
\end{lstlisting}
%
The following extra flags may be added to step 2 to fine-tune 
configuration.
%
\begin{lstlisting}[style=BashStyle]
-DCMAKE_BUILD_TYPE=DEBUG|RELEASE|BIT --- Optimisation level
    DEBUG:   No optimisation
    BIT:     Maximum optimisation while remaning bit-reproducible
    RELEASE: Maximum optimisation
-DENABLE_OMP=OFF --- Disable OpenMP
-DENABLE_MPI=OFF --- Disable MPI
-DENABLE_FORTRAN=OFF --- Disable Compilation of Fortran bindings
\end{lstlisting}
%
%
\begin{notebox}
By default compilation is done using shared libraries. Some systems have 
linking problems with static libraries that have not been compiled with 
the flag \inltc{-fPIC}. In this case, also compile atlas using static 
linking, by adding to step 2 the flag: \inltc{-DBUILD\_SHARED\_LIBS=OFF}
\end{notebox}
%

\section{Runtime Configuration}

\Atlas behaviour can be configured through variables defined at the command-line, 
in the environment, or in configuration files. In the following table, the column 
'variable' can be edited in configuration files.

\begin{table}[htb!]
\begin{center}
\begin{tabular}{l|c|c|r}
\toprule
variable & command line & environment & default \\ 
\midrule
  \inltc{debug} 					& \inltc{--debug} 
& \inltc{\$DEBUG} 					& \inltc{0}   			\\  
									& \inltc{--name}  	
&                   				& \inltc{<app>} 		\\
  \inltc{<name>.configfile} 		& \inltc{--conf} 
& \inltc{<name>\_CONFIGFILE} 		& \inltc{<name>.cfg}	\\ 
  \inltc{<app>.configfile} 			& 
& \inltc{<app>\_CONFIGFILE} 		& \inltc{<app>.cfg}		\\ 
  \inltc{atlas.configfile} 			& \inltc{--atlas\_conf} 
& \inltc{atlas\_CONFIGFILE} 		& \inltc{atlas.cfg}   	\\ 
  \inltc{atlas.logfile} 			& \inltc{--logfile} 
& \inltc{atlas\_LOGFILE} 			&    					\\
  \inltc{atlas.logfile\_task}  		& \inltc{--logfile\_task} 
& \inltc{\$ATLAS\_LOGFILE\_TASK} 	& \inltc{-1} 			\\
  \inltc{atlas.console\_task}    	& \inltc{--console\_task} 
& \inltc{\$ATLAS\_CONSOLE\_TASK}  	& \inltc{0}				\\
  \inltc{atlas.gmsh.surfdim}    	&                  
&                            		& \inltc{2}           	\\
  \inltc{atlas.gmsh.gather}    		&                  	
&                          			& \inltc{false}        	\\
  \inltc{atlas.gmsh.ghost}     		&                  
&                          			& \inltc{false}         \\
  \inltc{atlas.gmsh.ascii}      	&                        
&  									& \inltc{true}         	\\
  \inltc{atlas.gmsh.elements}   	&    
&									& \inltc{true}         	\\
  \inltc{atlas.gmsh.edges}      	&                   
&                          			& \inltc{true}        	\\
  \inltc{atlas.gmsh.levels}     	&                
&                          			& \inltc{[]}        	\\
  \inltc{atlas.meshgen.angle}   	&                  
&                          			& \inltc{0}             \\
  \inltc{atlas.meshgen.triangulate}	& 
&                          			& \inltc{true}      	\\       
\bottomrule
\end{tabular}
\label{tab:runtime}
\caption{Runtime configuration variables}
\end{center}
\end{table}